\eucommentary{1-2 pages}
\eucommentary{\emph{Describe the specific objectives for the project,
which should be clear, measurable, realistic and achievable within the
duration of the project. Objectives should be consistent with the expected
exploitation and impact of the project (see section 2).
Desirable keywords: sustainability, impact, reproducibility,
interoperability, ...
}
}
\medskip
\noindent The aims of \TheProject are to:

\begin{compactenum}[\textbf{Aim} 1:]
\item Enable a
  \textbf{sustainable}, \textbf{community-developed}, general purpose, \textbf{interoperable} toolbox for
  interactive computing, data processing, and visualization
  \textbf{that facilitates the entire life-cycle of Open Science},
  from initial exploration to \textbf{reproducible publication}, research and development in
  industry, teaching, and outreach.
  This is by supporting and steering the Jupyter software ecosystem,
  which exists to develop open source software,
  open standards, and services for interactive computing across dozens of programming languages.

\item Leverage this technology for all scientists, across borders,
  domains, disciplines, and demographics, through
  \textbf{free public distributed collaborative services} tightly integrated
  into the European Open Science Cloud (EOSC),
  in collaboration with a federation of related services
  operated by the wider community.

\item Demonstrate the value and versatility of such services through
  \textbf{innovative co-designed tailored applications} in a variety of disciplines and
  contexts.

\item \textbf{Support Open Science} and maximize impact through development and
  dissemination of best practices,
  \textbf{training}, and \textbf{community building}
  around the usage and development of the above toolbox,
  with a focus on \textbf{reproducibility} and
  \textbf{interoperability}.
\end{compactenum}
% develop and support the Jupyter ecosystem in a direction that benefits and facilitates open science
%     make these tools accessible to as many people as possible via operation of free, public services
%     demonstrate and ensure that these developments are useful to real scientists and the public
%     foster open science through training of students and researchers in best practices using these tools


% \item \label{aim:facilitation}
%   Facilitate Open Science through the development
%   of tools enabling reproducibility, sharing, and collaboration.

% \item \label{aim:accessibility}
%   Maximise accessibility and interoperability of Open Science services and tools,
%   across domains, disciplines, and demographics.

% \item \label{aim:sustainability}
%   Maximise sustainability of software tools for Open Science
%   by developing the community and contributing
%   to and supporting community-led software efforts.



%   Support open source software for open science, and notably the
%   Jupyter ecosystem,



% \end{compactenum}

\medskip
\noindent We will achieve our aims through the following objectives:

\begin{compactenum}[\textbf{Objective} 1:]

\item \label{obj:deployment}
  \textbf{Infrastructure and services for Jupyter on EOSC} ---
  Contribute a distributed infrastructure to the European Open Science Cloud
  (EOSC) and the wider Open Science community that can be tailored to
  provide a multitude of generic or specialized services that facilitates
  open science in a \textbf{wide range of scientific domains} and projects.
  This infrastructure will build on the Jupyter project and ecosystem,
  taking the form of a federation of JupyterHub/Binder instances,
  tightly integrated into the EOSC-Hub.
  To maximize \textbf{impact, outreach, and sustainability},
  the federation will include and encourage instances operated by
  external partners, whether free or non free, public or private,
  general purpose or custom built for specific needs -- e.g.
  providing access to specialized or large data sets, or specific hardware.
  This objective will be supported by improvements to the Jupyter
  deployment toolbox which fosters \textbf{reuse and interoperability
    beyond the Jupyter ecosystem}.

\item \label{obj:interactivity}
  \textbf{Improving interactive computing} ---
  Improve the interactive computing capabilities of
  students, researchers, educators, and the public
  through contributions to the Jupyter environment,
  in the form of developments of interactive widgets,
  visualization tools, collaboration features, dashboards,
  teaching tools,
  and expanded support for more language communities,
  such as interactive C++.
  While Jupyter is already widely used,
  there are many areas
  of interactive exploration that can be developed further.

\item \label{obj:reusability}
  \textbf{Reproducibility and FAIR data} ---
  Extend facilities for
  \textbf{reproducibility of computational environments}
  and facilitating \textbf{FAIR data practices}.
  We will contribute to the recording and reproducibility
  of environments with repo2docker and Binder,
  and extend capabilities to better support FAIR
  data requirements. In particular, the archival of execution
  environments to support \textbf{reusability} of notebooks in the future
  needs attention. Such notebooks may, for example, be published alongside
  traditional publications to detail the computation of published data
  and figures, and address the Re-usable requirement of FAIR data.

\item \label{obj:demonstrators}
  \textbf{Demonstrators in science and education} ---
  We will demonstrate and ensure the versatility and value of the components and
  the services built from them,
  through applications to a number of
  domains in academic research, education, research infrastructures, SMEs, and for
  the public sector, driven through our project partners. In
  particular, we will contribute demonstrators in the following areas:
  astronomy (\taskref{applications}{astro}), education
  (\taskref{applications}{teaching}), fluid dynamics
  (\taskref{applications}{application-gpu}), geosciences
  (\taskref{applications}{geoscience}), health
  (\taskref{applications}{opendose-analysis}), mathematics
  (\taskref{applications}{math}),
  and photon science (\taskref{applications}{reproducibility-xfel}),
  involving universities, research infrastructure facilities, and SMEs.

\item \label{obj:outreach-and-engagement}
  \textbf{Outreach, engagement, and sustainability} ---
  Reach out to scientists and the wider Open Science and Open Data
  communities to encourage engagement
  and exploitation of the EOSC-Hub and the Jupyter-based Open Science
  Services for their research domains and interests.
  Engaging a larger community will help \textbf{ensure the sustainability} of
  the services and underlying infrastructure by distributing its
  development, hosting, and maintenance over stakeholders from a
  variety of institutions and backgrounds,
  from the private sector to public research, education
  and open government.

\end{compactenum}

\begin{table}
  \label{tab:objectives-tasks}
  \caption{
  Each objective and the tasks which further that goal.}
  \begin{tabular}{|m{.3\textwidth}|m{.7\textwidth}|}

    \hline

    \textbf{Objective} & \textbf{Tasks}
    \\\hline

    \ref{obj:deployment} &

    \longtaskref{core}{maintenance}
    \longtaskref{core}{jh-bh-conv},
    \longtaskref{eosc}{eu-binder},
    \longtaskref{eosc}{eosc},
    \longtaskref{eosc}{jh-bh-deployment}

    \\\hline

    \ref{obj:interactivity} &

    \longtaskref{core}{accessibility},
    \longtaskref{core}{collaboration},
    \longtaskref{ecosystem}{xeus-cpp},
    \longtaskref{ecosystem}{jupyter-widgets},
    \longtaskref{ecosystem}{teaching-tools}

    \\\hline

    \ref{obj:reusability} &

    \longtaskref{ecosystem}{r2d-and-binder},
    \longtaskref{ecosystem}{reproducibility}

    \\\hline

    \ref{obj:demonstrators} &
    \longtaskref{applications}{astro},
    \longtaskref{applications}{teaching},
    \longtaskref{applications}{application-gpu},
    \longtaskref{applications}{geoscience},
    \longtaskref{applications}{opendose-analysis},
    \longtaskref{applications}{math},
    \longtaskref{applications}{reproducibility-xfel}

    \\\hline

    \ref{obj:outreach-and-engagement} &

    \longtaskref{education}{workshops},
    \longtaskref{education}{online-resources},
    \longtaskref{education}{helpdesk}

    \\\hline

  \end{tabular}
\end{table}
