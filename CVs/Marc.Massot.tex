\begin{participant}[type=R,PM=2,gender=male]{Marc Massot}

    Marc Massot obtained his PhD in Applied Mathematics from Ecole Polytechnique, 
    France, in 1996. After a
    year at Yale University, Department of Mechanical Engineering, he obtained a
    CNRS position in the Applied Mathematics Laboratory of the University of Lyon,
    France, where he stayed until 2005. He was offered an Associate Professor
    position at Ecole Centrale Paris when he installed a mathematics team in the
    EM2C mechanical engineering laboratory. From 2008 to 2010, he had the responsibility of 
    structuring scientific computing at INSMI (French National Mathematics Institute of CNRS) 
    at a national level in the French Mathematical Community. He was Visiting Professor 
    for one year at the Center for Turbulence Research, Stanford University, in 2011-2012 
    and created and chaired the Fédération de Mathématiques de l’Ecole Centrale Paris
    between 2013 and 2016. He initiated the Computing Center (Mésocentre) of Ecole
    Centrale Paris in 2010, of which he was the deputy director until 2016 and he
    is scientific adviser at ONERA DMPE and scientific collaborator at Maison de la
    Simulation since 2013. Full Professor since 2011, he has been recruited on a
    full Professor position at Ecole Polytechnique, Centre de Mathématiques
    Appliquées in 2017 and co-chairs the Initiative HPC@Maths, which foster the interaction of mathematics, 
    scientific computing and HPC with industry and in particular SMEs. 
    
    His main fields of
    research are mathematical modeling and numerical analysis, analysis of PDEs and
    dynamical systems for multi-scale systems, scientific computing and high
    performance computing with applications in combustion, two-phase flows, plasma
    physics and biomedical engineering. 
    
    Since his arrival at Ecole polytechnique, he has been very involved 
    in promoting the Jupyter project, creating courses entirely relying on 
    Jupyter notebooks (such as "Dynamical systems for the modeling and simulation 
    of multi-scale reacting media" MAP551) and making the link with the various departments
    and students of Ecole polytechnique, as well as Paris-Saclay community. 
    One of the leaders of the new Computing Center
    in the process of being created, he is also involved with the direction of Ecole polytechnique
    for the project of building an infrastructure for Jupyter at the level of the school, for its use in 
    both research and teaching.

\end{participant}