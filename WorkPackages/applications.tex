\TOWRITE{ALL}{Proofread WP 1 Management pass 1}
\begin{draft}
\TOWRITE{PS (Work Package Lead)}{For WP leaders, please check the following (remove items
once completed)}
\begin{verbatim}
- [ ] have all the tasks in this Work Package a lead institution?
- [ ] have all deliverables in the WP a lead institution?
- [ ] do all tasks list all sites involved in them?
- [ ] does the table of sites and their PM efforts match lists of sites for each task?
      (each site from the table is listed in all relevant tasks, and no site is listed
      only in the table or only at some task)
\end{verbatim}
\end{draft}

\begin{workpackage}[id=applications,wphases=0-48,swsites,
  title=Applications and showcases,
  short=Applications,
  lead=XFEL,
  % EGIRM=4,
  CDSRM=12,
  INSERMRM=24,
  QSRM=6,
  SILRM=12,
  SRLRM=9,
  UIORM=12,
  UPSUDRM=20,
  WTTRM=3,
  XFELRM=36,
  EPRM=3,
]
\begin{wpobjectives}
  The objectives of this work package are
 \begin{compactitem}
   \item to guide the development of core tools by simultaneously
     developing and using applications in diverse fields with active
     scientists from these fields, and
   \item demonstrate that the tools we develop are valuable to diverse
     fields of science, thus ensuring we develop e-infrastructure and
     services which can cater for a broad customer base of EOSC.
 \end{compactitem}
\end{wpobjectives}

\begin{wpdescription}

  In order to ensure that the work we do is beneficial to science and
  society, we will develop applications in diverse domains, using the
  components developed in work packages \WPref{core} and
  \WPref{ecosystem}.

  By developing applications simultaneously with the core
  infrastructure and services, we ensure that we are serving
  real-world use cases with our development.  Feedback from
  applications development can guide development of core features.  In
  addition to helping ensure that our work is useful to the chosen
  application domains, the applications serve as demonstrations of
  this utility.

  One of the key aspects of Jupyter that make it the choice for
  \TheProject, is that it is generic, based on open protocols and
  widely useful in diverse domains, from physics to life sciences to
  education and the global community working with public and private
  data. We have selected a number of applications in a variety of domains
  to demonstrate the broad impact of \TheProject, in particular in the
  areas of education (\localtaskref{teaching}), photon science and
  imaging (\localtaskref{reproducibility-xfel}), astronomy
  (\localtaskref{astro}), geosciences (\localtaskref{geoscience}),
  mathematics (\localtaskref{math}) and health (\localtaskref{opendose-analysis}).

\end{wpdescription}

\begin{tasklist}
% add tasks from task directory here
% % template for a task
% each task should be added to exactly one workpackage
% in the workpackage task list
\begin{task}[
  title=Task title,
  id=task-id,
  lead=XXX,
  PM=1,
  wphases={0-48},
  % don't include lead here
  partners={SRL,XFEL}
]
  The task includes the following activities
  \begin{compactitem}
  \item ...
     % deliverable will be defined in the appropriate WorkPackage.tex
    (\localdelivref{deliv-id})
  \end{compactitem}
\end{task}

\begin{task}[
  title=Demonstrator: enriched teaching with Jupyter,
  id=teaching,
  lead=EP,
  PM=6, % EP: 3PM, UPSUD: 2PM, EuXFEL 1 PM
  wphases={0-48},
  partners={EGI,UIO,UPSUD,XFEL}
  ]


  In this task, we will \textbf{deliver and help deliver Jupyter-based
    courses at a large scale} in our own institutions, as a mean to
  \textbf{inform, evaluate, provide feedback on, and demonstrate the
    value of the work performed} in this project in the context of
  higher education, as well as to \textbf{develop and share best
    practices} and \textbf{demonstrate and disseminate} Jupyter's full
  potential for teaching.

  \'Ecole polytechnique and Université Paris-Sud are particularly well
  suited for this task because they
  \begin{enumerate}
  \item host a variety of local infrastructure (dedicated servers,
    local cloud, computer labs, ...);
  \item host a reactive community with highly qualified research
    software engineers (DevOps, software developers), researchers,
    professors, and students that have been working together on this
    topic for several years, with close collaboration between the two
    sites;
  \item offer a diversity of courses, in many disciplines and ranging
    from large lower undergraduate courses to specialized classes for
    graduate students and top notch engineers;
  \item have strong support from their teaching departments.
  % \item have an influential alumni structure through which the
  %   technology will be propagated.\TODO{do we want this item?}
  % \item will create a reactive community of students and researchers,
  %   where the dissemination of the project tools and experience will
  %   be easy, which will organize Jupyter days as in
  %   2018\footnote{\url{http://www.cmap.polytechnique.fr/~massot/Personal_web_page_of_Marc_Massot/JupyterX.html}}
  %   on a regular basis.
  \end{enumerate}

  % \TODO{HF: Loic, can you complete this, please?}

  % A variety of courses are delivered at Université Paris Sud using
  % Jupyter technologies. This includes for example programming classes
  % in C++ at lower undergraduate level (400 students per year since
  % 2017), a series of undergraduate and graduate math courses (computer
  % aided mathematics, computer algebra, numerical methods), or courses
  % in physics, bio-informatics, etc. To support these courses, a
  % JupyterHub service has been deployed in 2017 and progressively
  % improved since, on Paris Sud's local cloud infrastructure, enabling
  % students and teachers to work from anywhere and any device.

  % The Mathematics department at Ecole polytechnique has started a reform of their various teaching offer based on
  % Jupyter for two years and several courses of the Bachelor program, 2nd and 3rd
  % year of Engineering school and Master program have already begun relying on a
  % strong use of Jupyter notebooks / JupyterHub\footnote{MAP551 - 2nd yer course
  % MAP411 - AMS X02 - Mooc INRIA preciser?} and this will continue with a strong
  % support of the Dean of undergraduate studies and of graduate studies. Besides
  % several software and research engineers in applied mathematics have been
  % recruited and participate in this effort, as well as a administration engineer
  % in order to help in terms of building an infrastructure dedicated to Jupyter
  % with the support of the head of the Ecole polytechnique in order to disseminate
  % the effort into various other departments (Physics, Mechanical Engineering,
  % Biology...), where already some courses are starting based in Jupyter. The
  % link with the computer science club of students of Ecole polytechnique (Binet
  % R\'eseau) has also been created with the project and a community is emerging.

  The task includes the following activities
  \begin{compactitem}
  \item Reinforce the use of Jupyter technology in Courses at
    all levels, notably in Mathematics and Data Science, in close
    collaboration between Ecole polytechnique and Université Paris Sud;
  \item Test the new developments and feed back to tasks
    \taskref{core}{jh-bh-conv}, \taskref{ecosystem}{xeus-cpp}, \taskref{ecosystem}{teaching-tools}, \taskref{applications}{math} and \taskref{eosc}{jh-bh-deployment};
  \item Follow up on a successful Jupyter day in
    2018~\footnote{\url{http://www.cmap.polytechnique.fr/~massot/Personal_web_page_of_Marc_Massot/JupyterX.html}}
    by organizing a yearly Jupyter event showcasing the latest
    advances for teaching and research;
  \item Foster sharing of experience, best practices and course
    material, at the local level, and then worldwide, through meetups,
    blogs, etc.
  \end{compactitem}
  %\TODO{update for consistency: report in the general WP report + a demonstrator?}
  The outcome of this task will be reported on in
  \delivref{applications}{applications-report} and selected
  teaching materials will be made available for interactive execution
  through our Binder service (\delivref{applications}{teaching}).
\end{task}

% template for a task
% each task should be added to exactly one workpackage
% in the workpackage task list
\begin{task}[
  title=Reproducible X-ray crystallography workflows at European XFEL,
  id=reproducibility-euxfel,
  lead=XFEL,
  PM=36,
  wphases={0-48},
  partners={XFEL}
  ]

  \TODO{which other partners are interested?}
  There is stiff competition for beam time to do research at European XFEL,
  and it is important to the 12 countries \TOWRITE{}{How many are EU member states?}
  which fund the facility that the best possible use is made of the data collected.
  All of the data will be made freely available after an embargo period of
  three years, and we are keen to complement this with reproducible data
  analysis, to confirm conclusions drawn and to allow re-analysis with new
  tools or for new purposes. \TODO{Cite data policy}
  If the analysis steps are not carefully recorded, there is a risk that the
  necessary understanding of the data is lost by the time it is made public,
  greatly reducing its scientific value.

  A major category of experiments already begun at European XFEL revolve around
  X-ray crystallography, using X-ray scattering to resolve molecular structures.
  Processing the data collected to produce meaningful structures is a
  significant computational task, and researchers have developed complex
  software for this, such as CrystFEL, Cheetah and CCTBX.
  % How do I properly cross-reference another task or workpackage?
  % using \WPref or \taskref.
  As an application of the reproducible workflows task in WP2, we will build
  workflows around such software, aiming to allow 'one button' replication
  of published structures once the raw data is made available.
  As part of European XFEL's data policy, all data is to be publicly available
  ?? years after it is collected.
  The task includes the following activities:
  \begin{compactitem}
  \item ...
    (\localdelivref{deliv-id})
  \end{compactitem}

  \TODO{Thomas, Hans, lots of work to be done here.}
\end{task}

\begin{task}[
  title=Astronomy application,
  id=astro,
  lead=CDS,
  PM=18,
  wphases={18-42},
  partners={QS,WTT,SRL,INSERM,XFEL}
]

  Context: The Strasbourg Astronomical Data Center (CDS) is scientific data 
  center hosted by the Observatory of Strasbourg. The CDS plays a unique and 
  essential role in astronomy by adding value to published and reference data. 
  CDS runs astronomical services that
  provide data for the world-wide astronomy research community. Its three main
  services (SIMBAD, VizieR and Aladin) are heavily used with up to one million
  queries per day.  These services be accessed through web interfaces, mainly
  for human interaction, as well as through programmatic interfaces, including
  the standardized protocols defined by the International Virtual Observatory
  Alliance.

\begin{figure}[ht!]\centering
  \includegraphics[width=0.6\textwidth]{python-astro-citations}
  \caption{Mentions of programming languages in refereed Astronomy papers, extracted from ADS. Python usage has increased dramatically in the recent years.}\label{fig:python-astro-citations}
\end{figure}

  Python and notebooks are rapidly increasing in importance for astronomy 
  research. Indeed, Python for Astronomy software ecosystem has known a 
  constant steady growth in the latest years, as shown in 
  figure~\ref{fig:python-astro-citations}. As Python and notebooks integrate 
  well together, the Jupyter notebook as an analysis tool is becoming a hot 
  topic in the astronomical world: large surveys like the LSST (Large Synoptic
  Survey Telescope) have endorsed the usage of the Jupyter platform for their 
    data access portal \cite{lsst2017scienceplatform}.\\


  We will develop a Jupyter-based framework to efficiently access, explore,
  visualize and analyze reference data that are available through CDS services 
  as a real example of using open astronomy data.
  We will provide scientific users with a set of customizable Jupyter notebooks
  for visualization and analysis tasks, providing a new level of
  interoperability with python libraries and notebooks as is highly demanded
  by the astronomy research community.

  The focus is on the two following user stories:
    \begin{compactitem}
        \item analysis of catalogue data results, up to billions of rows.
              Tabular data is the typical output of SIMBAD and VizieR data.
        \item modular dashboard-like interface providing a top level
              interactive view of the available data for a given astronomical
              object and enabling loading and analysis of those data.
    \end{compactitem}

  This task will build on existing Python libraries to access CDS data
  (\textit{astroquery.[cds/simbad/vizier/xmatch]}). For visualization, we will 
  use proven tools like \textit{GLUE} and \textit{ipyvolume}, which are now 
  built upon the Jupyter stack.
  We will also make significant improvement to existing Jupyter widgets 
  (\textit{ipyaladin}, interactive sky atlas running in the notebook) and 
  develop a new widget to offer a tree-like view of available datasets.

  We will also develop Python libraries to allow integration and usage in
  notebook of existing CDS infrastructure services, namely CDSLogin (which
  provides authentication) and CDS MyData (remote storage space for tabular
  data).
  This will allow the user to interact with one's personal storage space from
  the notebook. It will also allow for advanced customisation of the interface 
  to fit user needs.

  The work is organised with a 2 stage approach. Firstly, the generated 
  notebooks will run locally on user machines (representing a milestone for 
  this task). Following the Binder development in \WPref{eosc}, we will aim 
  to run these notebooks on the European Binder Service. The aspiration is 
  that this contributes to the development of innovative services for the EOSC.
  The deliverable of this task will be a demonstrator available to the 
  scientific user community (\localdelivref{application-astro}).

  By milestone 3, astronomical data services based on reference astronomy data from CDS are made available in Jupyter notebooks, and decision point on how to use developments of WP5 for running these notebooks on the European Binder Service.

\begin{figure}[ht!]\centering
  \includegraphics[width=0.6\textwidth]{astro-aladin-snapshot}
  \caption{Simbad objects, XMM and Hubble coverages overlaid on Digital Sky Survey imagery in the vicinity of the Horsehead nebula, and visualized in Aladin Desktop software.}\label{fig:astro-aladin-snapshot}
\end{figure}

  Access to the notebooks will be provided as a one-click action option from
  SIMBAD and VizieR results pages.
  Thus, providing with a one-click way of visualizing, filtering and analyzing
these potentially large tables will bridge the gap between access and analysis
of the data, with zero installation for the user.
  For specific science cases, we will explore rendering of notebooks with 
  interactive widgets through "Voila", as to allow users not familiar with 
  Python to benefit from the Jupyter notebook framework.
  Figure~\ref{fig:astro-aladin-snapshot} depicts typical data objects we want to analyse and interact with in the notebooks: images, catalogue data, datasets coverages.

  These new developments will be highly visible to the large number of astronomers who use the CDS services (50,000 unique visitors per month) and such tools are in high demand by these users.

  The CDS expertise in astronomy data and interfaces will be profitably combined with expertise of BOSSEE partners to ensure the deployment of high quality widgets (Simula, WildTree Tech, QuantStack).



Deliverable: \localdelivref{application-astro}


\end{task}

\begin{task}[
  title=Demonstrator: Geosciences,
  id=geoscience,
  lead=UIO,
  PM=22,
  wphases={0-48},
  partners={EGI,QS,SRL,UPSUD}
]

% UPSud involvement: UPSud has a geoscience group (GEOPS) and will be
% interested in using the tools developed here. No formal PM.

This task aims at deploying a BinderHub for Big data geosciences and \emph{voila} innovative interactive \emph{App}. Co-design will take place from the beginning of \TheProject to tune the development of Jupyter ecosystem components and fulfill the need of the geoscience community.

\textbf{Visualization}

\begin{compactitem}
  \item Improvement upon existing mapping tools for specialized
    visualization of in-situ and model-generated data arizing in
    specific use cases (Land, river-runoff, ocean, ice, wave and
    atmosphere models, particle dispersion models, oil spill models,
    etc.).

    This may take the form of additions to the \emph{ipyleaflet} and
    \emph{folium} extentions to JupyterLab, as well as the production of
    curated examples in the documentation of ipyleaflet addressing these
    specific use cases.

  \item Improvements of the tooling for 3-D visualization of
    geographical datasets in the Jupyter notebook, for use cases such as
    displaying volcanic plumes (injection of aerosols in the various
    atmospheric layers), the quasi biennial oscillation (inversion of
    the wind direction in the tropical stratosphere), atmospheric rivers
    (flowing column of condensed water vapour in the atmosphere) and
    also at smaller scales to visualize 3-D discrete particle simulations
    of sheared granular fault zones.
\end{compactitem}

\textbf{Collaboration with Jupyter with specialized tools for earth sciences}

\begin{compactitem}
  \item adding the ability to interactively integrate information or corrections
    observed during field trips, correspdonding to specific geographical locations.

  \TODO{Concurrent editing links to real-time editing from the
    core WP2 - mention link?}

  \item adding the ability to deploy Jupyter-based applications together with
    the correspdonding execution environment, both in the form of a runnable
    notebook with \emph{Binder} or as a read-only yet interactive \emph{Voila}
    dashboard.
\end{compactitem}

\textbf{Data processing tools}

\begin{compactitem}
  \item streamlining visualization of standard data formats such as \emph{NetCDF}
  with ipyleaflet, and ipyvolume.

  \item better support for \emph{geopandas} dataframes in Jupyter interactive
  visualization tools.
\end{compactitem}
\end{task}

\begin{task}[
  title=Demonstrator: Interactive Mathematics with Jupyter Widgets,
  id=math,
  lead=UPSUD,
  PM=15, % UPSUD:16, QS:1, EGI: 1
  wphases={0-36},
  partners={EGI,EP,QS}
  ]

  The aim of this task (see page
  \pageref{sec:concept-demonstrator-math} for context)
  is to build on this experience to further
  develop and promote the use of Jupyter widgets for interactive
  Mathematics. This will include the following actions:
  \begin{compactitem}
  \item Engage with the community through tutorials, workshops, online
    discussions, for codesign and for dissemination of the outcomes.
  \item Tackle hurdles to real-time interactivity, typically by
    modernizing the existing 2D and 3D visualization tools in
    SageMath. % E.g.: We don't use Matplotlib's integration in Jupyter
  \item Bring \software{sage-combinat-widgets} and
    \software{sage-explorer} from usable prototypes to standard tools,
    and further contribute to the development of the \software{Francy}
    framework.
  \item Develop other generic mathematical widgets according to the
    users popular requests.
  \item Demonstrate the value all of the above through applications in
    research and teaching.
  \item Publish selected demonstration notebooks for interactive use on
    \TheProject's EOSC services (\delivref{applications}{demonstrators}).
  \end{compactitem}
  The work carried out will be reported on in
  \delivref{applications}{applications-report}.
\end{task}

\begin{task}[
  title=Nuclear Medicine application,
  id=opendose-analysis,
  lead=INSERM,
  PM=24,
  wphases={0-24},
  partners={INSERM}
]
  % Scientific description
  Nuclear Medicine is a field of medicine where radioactive material
  (radiopharmaceutical) is used for diagnostic and therapy. Even though the
  majority of Nuclear Medicine procedures (90\% according to successive EANM
  surveys) are diagnostic examinations, therapeutic applications tend to
  develop and drag more and more attention, for example for the treatment of
  neuroendocrine tumours.
  
  The formalism used to objectively characterise the irradiation process is
  similar for both application types: it was introduced in the late 60s by the
  MIRD (Medical Internal Radiation Dose) committee of the American Society of
  Nuclear Medicine (SNM). This formalism requires two independent quantities;
  the radioisotope cumulated activity ($Bq.s$) in the source (tissue/organ) and
  the mean absorbed dose to a given target (tissue/organ) per radioisotope
  disintegration (S-value, $Gy.Bq^{-1}.s^{-1}$). The S-value calculation
  requires a clear definition of the geometry of the patient (or the model) and
  radioisotope decay characteristics, it can be expressed as a linear
  combination of yields/energies ($J$) and Specific Absorbed Fractions (SAF,
  $g^{-1}$).
  
  The calculation of SAFs involves radiation transport modelling and energy
  deposition scoring in anthropomorphic models, usually based on Monte Carlo
  simulation. Historically, SAFs were computed from mathematical models -
  simplistic approximations to human geometry. The advent of voxel-based
  computational models requires a new appraisal of dosimetric data. For
  example, the models recently proposed by the International Commission on
  Radiation Protection (ICRP) include 140 possible radiation sources, leading
  to around 20000 source/target combinations. The production of SAFs for these
  models for all possible source regions, radiation types and energies
  represent an important computation time (millions of CPU hours).
  
  The OpenDose project is a collaborative effort to generate reference
  dosimetric data using various Monte Carlo codes across different teams. The
  collaboration includes at the moment 14 research teams over 18 institutes.
  The idea is to impulse the collaborative development of a reference database,
  freely available, proposing dosimetric data applicable in a context of
  nuclear medicine dosimetry (for therapy and diagnostic applications). A major
  aspect of the project is the development of tools ensuring traceability and
  robustness of generated results.

  % Technical description
  OpenDose data is produced using the five most represented Monte Carlo
  simulation software in medical applications: Geant4/GATE, MCNP, EGS, PENELOPE
  and Fluka. Each simulation consists of calculating radiation transport in
  anthropomorphic models for specific parameters (source organ, particle type,
  energy, model and number of primaries to simulate). Every simulation produces
  binary (3D matrices) and ASCII files for a total of $\sim$150MB / simulation.
  The 3D matrices contain energy deposited per voxels, and ASCII files contain
  pre-processed data corresponding to energy deposited per regions such as
  organs and tissues. These raw outputs are later processed into dosimetric
  data such as Specific Absorbed Fractions (SAFs) and S-values.
  
  Producing data for one model (ex. adult female) requires $\sim$30 000
  simulations, with the workload shared between the different teams and
  software.
  
  The data produced by all the teams is currently centralised at CRCT,
  processed and fed into a local SQL database. 

  This collaborative effort rise some challenges:
  \begin{compactitem}
  \item Data production: a total of 750 K hours of CPU time is needed per
    model.
  \item Volume of data: one model represents TB of raw data that can be
    heterogeneous from the different teams.
  \item Data analysis: raw data has to be processed into dosimetric data in a
    robust and reproducible way.
  \item Database: has to be efficient and handle all the data (raw and
    processed).
  \item Visualization: display and compare results from all teams.
  \end{compactitem}

  The objective of this task is to build on the Jupyter ecosystem to create a
  unified data analysis framework for the OpenDose project. By building a set
  of tools to access and process data, this will ensure the production of
  traceable and reproducible dosimetric data for the OpenDose project members.

  Another major aspect of the OpenDose collaboration is to provide an open
  access to the generated dosimetric data. For that purpose a website is under
  development to allow data download and simple dosimetry calculations. For
  users who need more advanced calculations, a dedicated Jupyter workspace will
  provide a set of tools to easily access, process and display the OpenDose
  data.

  The task includes the following activities:
  \begin{compactitem}
  \item Developing tools to work seamlessly on the SQL database.
  \item Developing data analysis tools.
  \item Developing visualization tools.
  \item Comparing results between teams.
  \item Disseminating results.
  \item Providing support to users.
  \end{compactitem}
  These developments will be integrated in the demonstrator
  (\localdelivref{opendose-analysis})

\end{task}

\begin{task}[
  title=Application: Visualisation and control of fluid dynamics in Jupyter notebook,
  id=application-gpu,
  lead=SIL,
  PM=13,
  wphases={4-36},
  % don't include lead here
  partners={EGI}
]

\textbf{Context}

In recent years, the lattice Boltzmann method (LBM) emerged as an
interesting alternative to more established methods for fluid flow
simulations. Sailfish-cfd \cite{januszewski2014sailfish} is an open
source implementation of the LBM on General Purpose Graphical Processing
Unit (GPGPU) devices. It is written in Python with real-time
generation of CUDA-C code.  In order to harvest capabilities of GPGPUs
one needs to access the specialized hardware, which usually is
available to researchers as remote HPC resources.  The typical fluid
dynamics research workflow consists of three stages: preparing
boundary conditions, running a simulation, and data analysis. The
first and last stage require capable and responsive user interface for
maniputation and inspection of 3d data.  The Jupyter 3d visualisation
widgets developed in \taskref{ecosystem}{jupyter-widgets} can fulfil
such needs.

Based on previous experience with K3D-jupyter\cite{K3D}
widgets we know that web browser based software can display moderate
dataset during the simulation. As the dataset is becoming larger the
visualisation in the browser turns out to be nontrivial due to
limitations of the browser itself and required large data transfers. It is
an open question how much of data processing should be performed on
server-side and what can be done on the client hardware (i.e. in the
widget in the browser side of the user). Our
experience suggests that there is no clear answer and it depends on
the size of the data and its nature. For example, volume rendering
technique can be very effective on the browser side but infers large data
transfers. One can perform it the server-side, in a distributed way if
the simulation uses many nodes, but the interactivity is limited by
network latency. We will attempt to provide practical
solutions to this issue.
%


\textbf{This task}


In this task, we will contruct tools for editing and inspecting
boundary conditions. Having such tools as Jupyter widgets will allow
to complete the workflow without leaving Jupyter notebook. We plan the
following activities
\begin{compactitem}
\item Development of Jupyter notebooks using fluid
  simulation based on the high-performance Sailfish-cfd solver.
\item Implementing advanced widgets for data visualisation of large
  fluid dynamics simulations.
\item Implementing widgets for inspection and editing boundary
  conditions in LBM.
\end{compactitem}

This work will closely interact in with the task
\taskref{ecosystem}{jupyter-widgets}: it will both provide guidelines
for the development to \taskref{ecosystem}{jupyter-widgets} and serve
as test case for implemented features in
\taskref{ecosystem}{jupyter-widgets}.

The deliverable of this part will be a demonstrator
(\localdelivref{lbm-jupyter}) available via the EOSC hub, and
contributions to report \localdelivref{applications-report}.

\end{task}

\end{tasklist}



\begin{wpdelivs}
%\TODO{update due date and startup!}
\begin{wpdeliv}[due=1,miles=startup,id=opendose-analysis,dissem=PU,nature=DEM,lead=INSERM]
  {Jupyter services for nuclear medicine dosimetry with the OpenDose project}
\end{wpdeliv}
\begin{wpdeliv}[due=42,miles=community,id=application-astro,dissem=PU,nature=DEM,lead=CDS]
    {Demonstrator of astronomical data services based on on reference astronomy data from CDS in jupyter notebooks is made available for use by the astronomy research community.}
\end{wpdeliv}
\begin{wpdeliv}[due=45,miles=final,id=xfel-workflows,dissem=PU,nature=DEM,lead=XFEL]
  {Demonstrator reproducible photon science}
\end{wpdeliv}

% \TODO{update milestone!}
\begin{wpdeliv}[due=36,miles=final,id=math,dissem=PU,nature=R,lead=UPSUD]
  {Report on Interactive Mathematics with Jupyter Widgets}
\end{wpdeliv}
\begin{wpdeliv}[due=48,miles=final,id=applications-report,dissem=PU,nature=R,lead=XFEL]
  {Evaluation of demonstrators and case studies. Report on
    feasibility, user feedback, potential shortcomings and
    improvement, to guide EOSC service design.}
\end{wpdeliv}
\begin{wpdeliv}[due=48,miles=final,id=gpu-jupyter-notebooks,dissem=PU,nature=R,lead=SIL]
  {Notebooks demonstrating SDE based research in Juyter notebook}
\end{wpdeliv}

\end{wpdelivs}
\end{workpackage}
%%% Local Variables:
%%% mode: latex
%%% TeX-master: "../proposal"
%%% End:

%  LocalWords:  workpackage wphases wpobjectives wpdescription pageref wpdelivs wpdeliv
%  LocalWords:  dissem mailinglists swrepository final-mgt-rep compactitem swsites ipr
%  LocalWords:  TOWRITE tasklist delivref
