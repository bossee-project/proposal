\begin{task}[
  title=Geosciences application,
  id=geoscience,
  lead=UIO,
  PM=24,
  wphases={0-48},
  partners={UIO,QS,SRL}
]


% Scientific description

The Geoscience field covers a wide range of disciplines, from deep mantle processes to atmospheric sciences, and all these activities are 
actually studied at the Department of Geosciences of the University of Oslo. In this regard, being the widest ranging earth sciences 
department in Norway, it is a unique “living” laboratory for the BOSSEE project. Some of the research groups are still virtually in the 
“stone age” in terms of digitization, and as far as can be from open and reproducible science. At the other extreme, other groups already 
had to make significant steps towards reproducible science with the use of complex numerical tools and huge amount of data necessarily 
shared within their community through National/International computing and storage infrastructures.

Two transversal and strategic research initiatives, namely LATICE (Land-Atmosphere Interactions in Cold Environments) and EarthFlows (Interface Dynamics in Geophysical Flows) intertwine these groups, although technical issues hinder the full exploitation of these multidisciplinary aspects.

% Technical description of tasks
The goal of this task is to select, develop and adapt Jupyter ecosystem tools to facilitate transversal research in geoscience, and will include activities around:

  \begin{compactitem}
  \item Mapping tools for visualization of in-situ and model generated data (Land, river-runoff, ocean, ice, wave and atmosphere models, particle dispersion models, oil spill models, etc.)  at high-latitudes and polar regions

  \item 3-D visualization for instance to display volcanic plumes (injection of aerosols in the various atmospheric layers), the quasi biennial oscillation (inversion of the wind direction in the tropical stratosphere), atmospheric rivers (flowing column of condensed water vapour in the atmosphere) and also at smaller scales to visualize 3D discrete particle simulations of sheared granular fault zones
   \item “Laboratory” notebooks with concurrent editing, annotation and easy integration of information/correction observed during field trips
   \item One-click button for reproducing, cloning and archiving workflows to support multidisciplinary activities
   \item Sharing read-only notebook-based interactive dashboards for novices

    (\localdelivref{geoscience})
  \end{compactitem}

All these developments will be used both for scientific research and in the classroom for teaching master's students with best practices in open science.

The Geosciences use case will make it possible to demonstrate the effectiveness of the BOSSEE co-design approach. The main challenges will 
be on the one hand to learn to take advantage of feedback given by users (either novices or experts) and on the other hand for BOSSEE 
developers to adapt their communication and language as well as their offer to the Geoscience. This will lead the way for on-boarding new 
communities. 
\end{task}
