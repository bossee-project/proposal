\begin{sitedescription}{UIO} \label{desc:UIO}

\begin{center}
\includegraphics[height=3cm]{Participants/Logos/UiO.png}
\end{center}

The University of Oslo (UiO) is Norway's oldest institution for research and higher education, with 28,000 students and 6,000 employees. UiO has 8 faculties, 2 museums and several centres. In addition, UiO has 10 Norwegian Centres of Excellence,  is ranked as the world's 62nd university, and has had 5 Nobel prize laureates. UiO aims to become an international hub for the research-based integration of computing into science education and has financed a university-wide hosting service for Jupyter notebooks through JupyterHub  to introduce a computational aspect to all curriculum programs in all science disciplines from bachelor to postdoctoral studies.

The University of Oslo is a Silver Partner to \href{https://carpentries.org}{The Carpentries}, an international successful community driven project with Instructors, Trainers, Maintainers, helpers, and supporters who share a mission to teach foundational computational and data science skills to researchers.

The Department of Geosciences of the Faculty of Mathematics and Natural Sciences is the broadest geoscience research-based teaching environment in Norway, and covers a wide range of disciplines from deep mantle processes to atmospheric sciences. It is organised in five sections and an administrative unit and supports two main strategic research initiatives:

- Land-Atmosphere Interactions in Cold Environments (\href{https://www.mn.uio.no/geo/english/research/groups/latice/}{LATICE})

- Interface Dynamics in Geophysical Flows (\href{https://www.mn.uio.no/geo/english/research/groups/earthflows/}{EarthFlows})


 The geosciences department has several large research projects financed by \href{https://www.forskningsradet.no/en/Home_page/1177315753906}{The Research Council of Norway}, EU and Norwegian companies.

 The University of Oslo aims to manage research data according to international standards, such as the \href{https://www.uio.no/for-ansatte/arbeidsstotte/fa/forskningsdata/fair-data/index.html}{FAIR principles}\footnote{Findable, Accessible, Interoperable and Reusable}, and thereby support the development of a global research community in which research data is widely shared.
 Since November 2017, UiO’s policy follows the "open as standard" principle in respect of access to research data \cite{datapolicy-uio}.

\subsubsection*{Curriculum vitae}

% Curriculum of the personnel at this institution. This includes
% to-be-hired people for which there is a tentative candidate.

\begin{participant}[type=R,PM=24,gender=female]{Anne Fouilloux}
  % type is one of:
  % - leadPI: leader of the participating institution
  % - PI: Principal Investigator
  % - R: researcher?
  % Who is the coordinator is specified elsewhere

  % PM=YYY:
  % A fair evaluation of the number of months you will be
  % spending on this specific project along the four years.
  % Typical numbers:
  % - full time hired personnel: 48 months
  % - lead PI or proposal coordinator: 8-12 months
  % - PI: 4-5 months
  % - participant: 2-6 months

  % salary=ZZZ:
  % Approximate monthly gross salary (in term of total cost for the
  % employer). This is optional. If you are uncomfortable having this
  % information in a public file, you can alternatively send the
  % information to Eugenia Shadlova, or to your institution
  % leader/manager if he is willing to fill in himself the budget
  % forms on the eu portal.

  % The above information is used to fill in various tables in the
  % proposal file, and to evaluate the cost of the project for the
  % institutions.

  % You may remove all those comments.

  % About half a page of free text; for whatever it's worth, you may see
  % Nicolas.Thiery.tex for an example.

  \medskip PhD, is a highly experienced Research Software Engineer dedicated to supporting
  researchers towards the adoption of Open Science best practices.

  With a solid background in Computer Sciences, she worked in various application fields, including environmental sciences, Intelligent Transport Systems, High-Performance computing, bio-informatics, meteorology and Geosciences.

  She is currently working in the IT group of the department of Geosciences at the \href{https://www.mn.uio.no/geo/english}{University of Oslo} and holds a 25\% at the \href{https://neic.no}{Nordic e-Infrastructure Collaboration} (NeIC) where she is involved on the \href{https://neic.no/nicest/}{Nordic Collaboration on e-Infrastructures for Earth System Modeling} (NICEST) and \href{https://coderefinery.org}{CodeRefinery}\ref{desc:coderefinery} projects on Training and e-Infrastructure for Research Software Development. 

   Since 2015, Anne Fouilloux has been very active with \href{https://carpentries.org}{The Carpentries}, a diverse and global community of volunteers and she teaches foundational coding and data science skills to students and young researchers. She is a certified \href{https://carpentries.org/instructors/}{Carpentries instructor}, \href{https://carpentries.org/trainers/}{instructor trainer} and \href{https://carpentries.org/maintainers/}{maintainer}. She has volunteered to help build \href{http://www.carpentrycon.org/}{CarpentryCon 2020} a biannual conference for members of the global Carpentries community and people with similar interests. 

  She is a member of the core team of the \href{https://www.uio.no/english/for-employees/support/research/research-data/training/carpentry/}{Carpentry@UiO} and is leading the \href{https://uio-carpentry.github.io/studyGroup/}{studyGroup@UiO} where students and researchers at the University of Oslo are committed to sharing skills, experiences, and ideas around open science, open source, code, and community in research.
\end{participant}

%%% Local Variables:
%%% mode: latex
%%% TeX-master: "../proposal"
%%% End:


%\input{CVs/First.Last.tex}
%\input{CVs/First.Last.tex}
%\input{CVs/First.Last.tex}

% For other to-be-hired person, please include here something like:
% \begin{participant}[type=res,PM=3,salary=5900]{NN}
%  <a _short_ description of the qualifications of whom you want to hire>
% \end{participant}

\subsubsection*{Publications, products, achievements}

\begin{compactenum}
\item \href{https://annefou.github.io/jupyter_publish/}{Publication ready scientific reports and presentations with Jupyter notebooks}, Anne Fouilloux, Research Bazaar 2019, \href{https://zenodo.org/badge/latestdoi/163517733}{DOI 10.5281/zenodo.2548936}

\item \href{https://annefou.github.io/jupyter_dashboards/}{Reproducible Research with Interactive Jupyter Dashboards}, 2018, Ana Costa Conrado, Gladys Nalvarte, Benjamin Ragan-Kelley and Anne Fouilloux, Research Bazaar 2018, \href{https://zenodo.org/badge/latestdoi/114125668}{DOI 10.5281/zenodo.1168721}

\item \href{https://annefou.github.io/metos_python/}{Working with Spatio-temporal data in Python}, 2017, Anne Fouilloux, \href{https://zenodo.org/badge/latestdoi/96184802}{DOI 10.5281/zenodo.1165281}
\end{compactenum}

\subsubsection*{Relevant projects or activities}

\begin{compactenum}
\item \href{https://coderefinery.org}{CodeRefinery} \label{desc:coderefinery} (2016-2021): 


The goal of this project is to provide students and researchers with infrastructure and training in the necessary tools and techniques to create sustainable, modular, reusable, and reproducible software.
This is a project within the Nordic e-Infrastructure Collaboration (\href{https://neic.no}{NeIC}), an organisational unit under \href{https://www.nordforsk.org/en}{NordForsk}.
NeIC is a Platinium Partner to \href{https://carpentries.org}{The Carpentries}.

The result of this project is a set of software development e-infrastructure solutions, coupled with necessary technical expertise and extensive training and on-boarding activities, training material and best practices guides which together form a Nordic platform for research groups and institutes to develop a better collaboration on software and thereby to catalyze reproducible research and collaboration.
\newline
CodeRefinery training material is licensed under \href{https://creativecommons.org/licenses/by-sa/4.0/}{CC BY-SA 4.0} and code examples are \href{https://opensource.org/}{OSI}-approved \href{https://opensource.org/licenses/mit-license.html}{MIT license}.

The University of Olso is a CodeRefinery partner and will ensure the complementarity of the two projects thus avoiding potential fragmentation. \TheProject will benefit from all this experience as well as the estbalished network in the Nordic Countries and beyond to fully realize the potential of \TheProject EOSC services. 
\newline


\item Nordic Collaboration on e-Infrastructures for Earth System Modeling (\href{https://neic.no/nicest}{NICEST}, 2017-2019) \label{desc:nicest}:

This project aims at networking, intensifying existing collaboration, and facilitating 
joint work on very specific topics helping, for example, building up knowledge and
competency, and harmonising certain procedures concerning e-Infrastructure topics.

\item \href{https://uiohive.github.io/Hive/}{UiOHive} (2018-2019) \label{desc:uiohive}: 

UiOHive provides a vital and novel competence at the Union of Internet of Things (IoT), Microcontroller / Hardware development, Artificial Intelligence (AI) and Machine Learning, and Data Science to enhance and strengthen collaboration between domains at the application of the aforementioned technologies. and the disciplines with competence to further develop technologies.
The purpose of GEOHive is to establish a central knowledge hub, centered around individuals interested in utilizing IoT technologies, applying Artificial Intelligence and Machine Learning to data challenges, and sharing knowledge across relevant interdisciplinary domains.

\item \href{https://www.mn.uio.no/geo/english/research/groups/latice/}{LATICE} (Land-ATmosphere Interactions in Cold Environments, 2015-2022): 
LATICE aims to advance the knowledge base concerning land atmosphere interactions and their role in controlling climate variability and climate change at high northern latitudes.

\item \href{https://www.mn.uio.no/geo/english/research/groups/earthflows/}{EarthFlows} (Interface Dynamics in Geophysical Flows, 2015-2022): 
The dynamics of interface processes during flows on Earth, including the geosphere, the hydrosphere, the cryosphere, and the atmosphere, including the behavior of the complex interfaces separating ‘Fluid Earth’ from ‘Solid Earth’.

The goal for the EarthFlows project is to provide fundamentally new understanding of the dynamics of fluid-solid interfaces for a number of important geophysical systems.

\end{compactenum}

\subsubsection*{Significant infrastructure}

\begin{compactenum}

\item \href{http://www.uh-iaas.no/}{Infrastructure as a Service}: the University of Oslo is part of the Norwegian Cloud Infrastructure for Research and Education and provide researchers with compute and storage medium-size resources. These include multi-GPUs clusters for big data analysis. The department of Geosciences is heavily relying on this services both for teaching and research work.

\item \href{https://sigma2.no}{UNINETT Sigma-2}: UNINETT Sigma2 manages the national infrastructure for computational science in Norway and offers services in High Performance Computing (HPC) and Data Storage and data analysis (Research Platform as a Cloud Service). The services are organized into infrastructural activities, financed by the Research Council of Norway and the Sigma2 consortium partners, which are the universities in Oslo, Bergen, Trondheim and Tromsø.

Services are freely available to individuals and groups involved in  research and education at Norwegian universities and colleges, and other organizations and project funded with public money. Cost efficient development, procurement, coordination and operation of the national e-infrastructure for research and education is the main focus for Sigma2.

The Department of Geosciences (University of Oslo) has been granted access to over 2 petabytes and several millions of CPU hours on the Norwegian High-Performance computers.
\end{compactenum}

\end{sitedescription}
%%% Local Variables:
%%% mode: latex
%%% TeX-master: "../proposal"
%%% End:
