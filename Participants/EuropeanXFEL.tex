\begin{sitedescription}{EuXFEL}
  \label{sitedescription:euxfel}

% PIC:
% see: http://ec.europa.eu/research/participants/portal/desktop/en/orga

% See ../proposal.tex, section Members of the Consortium for a
% complete description of what should go there

  The European X-Ray Free-Electron Laser (EuXFEL) Facility GmbH is a
  limited liability company under German law. At present, 11 countries
  are supporting European XFEL through cash and in-kind contributions:
  Denmark, France, Germany, Hungary, Italy, Poland, Russia, Slovakia,
  Spain, Sweden, and Switzerland. The company is in charge of the
  construction and operation of the European XFEL, a 3.4$\,$km long
  X-ray free-electron laser facility extending from Hamburg to the
  neighbouring town of Schenefeld in the German federal state of
  Schleswig-Holstein. Civil construction started in early 2009; the
  beginning of user operation is planned for 2017. With its repetition
  rate of 27,000 pulses per second and a peak brilliance a billion
  times higher than that of the best synchrotron X-ray radiation
  sources, it is expected that the European XFEL will enable the
  investigation of still open scientific problems in a variety of
  disciplines (physics, structural biology, chemistry, planetary
  science, study of matter under extreme conditions and many others).

  European XFEL, a landmark on the ESFRI Roadmap, is a single site
  X-ray research infrastructure. When operational, 3~beamlines and
  6~experiments will be available for scientific users. The SASE1
  beamline comprises the instruments Single Particles, clusters, and
  Biomolecules and Serial Femtosecond Crystallography (SPB/SFX) and
  Femtosecond X-ray Experiments (FXE), SASE 2 includes Materials
  Imaging and Dynamics (MID) and High Energy Density Science (HED) and
  SASE3 Small Quantum Systems (SQS), and Spectroscopy \& Coherent
  Scattering (SCS).


\subsubsection*{Curriculum vitae}

% Curriculum of the personnel at this institution
%
\begin{participant}[type=leadPI,PM=4,gender=male]{Hans Fangohr}
  % type is one of:
  % - leadPI: leader of the participating institution
  % - PI: Principal Investigator
  % - R: researcher?
  % Who is the coordinator is specified elsewhere

  % PM=YYY:
  % A fair evaluation of the number of months you will be
  % spending on this specific project along the four years.
  % Typical numbers:
  % - full time hired personnel: 48 months
  % - lead PI or proposal coordinator: 8-12 months
  % - PI: 4-5 months
  % - participant: 2-6 months

  % salary=ZZZ:
  % Approximate monthly gross salary (in term of total cost for the
  % employer). This is optional. If you are uncomfortable having this
  % information in a public file, you can alternatively send the
  % information to Eugenia Shadlova, or to your institution
  % leader/manager if he is willing to fill in himself the budget
  % forms on the eu portal.

  % The above information is used to fill in various tables in the
  % proposal file, and to evaluate the cost of the project for the
  % institutions.

  % You may remove all those comments.

  % About half a page of free text; for whatever it's worth, you may see
  % Nicolas.Thiery.tex for an example.



  \medskip Hans Fangohr is an academic at the University
  of Southampton in the United Kingdom since 2002 (full professor
  since 2010), and leading the data analysis services at European XFEL
  in Germany since 2017.

  He has been a long term proponent of Open Science, and in particular
  involved with the the use and further development of the Jupyter
  Notebook to enable this. He has hosted Thomas Kluyver at the
  University of Southampton since 2015 from where he contributed as a
  core developer of the Jupyter team. As a PI in the EC-funded e-INFRA
  OpenDreamKit project (2015-2019), he has pushed forward the use of
  Jupyter Notebooks for reproducible computational science, and
  started the notebook validation tool (NBVAL). He made use of the
  Jupyter Ecosystem for research and education at graduate and
  postgraduate level at the University of Southampton, and shared
  resources widely, including a text book provided through Jupyter
  Notebooks, which can be executed interactively online [1].

  Since 2017, he is designing data analysis services and
  infrastructure at the European XFEL research facility. European XFEL
  is using IPython and the Jupyter Notebook as core utilities in their
  large scale experiment control, data capture and data
  analysis. Within the e-INFRA project PaNOSC (Photon and Neutron
  Science Open Cloud, 2018-2021), he is leader of the Work Package 4,
  which is focused on data analysis services for the EOSC Hub, and the
  use of the Jupyter notebook with its existing features on the EOSC
  hub.

  In this project (BOSSEE), where new capabilities for the Jupyter
  notebook and ecosystem are being designed, Hans' wide experience and
  interaction with different science groups will be beneficial to
  ensure the outcome is of value to open science in many domains. This
  includes him chairing the interdisciplinary computational modelling
  group at the University of Southampton (200 academics, 2008-2017),
  chairing the national EPSRC scientific advisory committee on High
  Performance Computing in the UK (2014-2017) and interacting with a
  large variety of science users at European XFEL in his role
  to lead the data analysis service provision.
\end{participant}

%%% Local Variables:
%%% mode: latex
%%% TeX-master: "../proposal"
%%% End:

%
\subsubsection*{Publications, products, achievements}

\begin{compactenum}
\item H.Fangohr, Python for Computational Science and Engineering
  (2018) DOI: 10.5281/zenodo.1411868 \newline
  https://github.com/fangohr/introduction-to-python-for-computational-science-and-engineering
\item H.Fangohr et al., “Data Analysis support in Karabo at European
  XFEL”, Proceedings of International Conference on Accelerator and
  Large Experimental Physics Control Systems 2017, ISBN 978-3-95450-
  193-9, Data Analytics, Barcelona, Spain, TUCPA01 (2017) DOI: 10.18429/JACoW-ICALEPCS2017-TUCPA01
\item H.Fangohr.
\emph{A Comparison of \software{C}, \Matlab and \Python as Teaching Languages in Engineering}
Lecture Notes on Computational Science \textbf{3039}, 1210-1217 (2004)
\end{compactenum}

\subsubsection*{Previous projects or activities}

\begin{compactenum}
\item OpenDreamKit (GA No. 676541) Open Digital Research Environment
  Toolkit for the Advancement of Mathematics, participant
\item EOSCpilot (GA No. 739563) The European Open Science Cloud for
  Research Pilot Project, participant
\item PaNOSC (GA No. 823852) Photon and Neutron Open Science Cloud, participant
\end{compactenum}

\end{sitedescription}



%KEY-MORE-TODOS



%%% Local Variables:
%%% mode: latex
%%% TeX-master: "../proposal"
%%% End:

%  LocalWords:  sitedescription Programme organisations programmes Centres subsubsection
%  LocalWords:  micromagnetic Nmag Fischbacher Franchin Bordignon Fangohr emph textbf
%  LocalWords:  Multiphysics summarised Iridis TFlops Modelling
