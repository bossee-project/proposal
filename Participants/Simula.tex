\begin{sitedescription}{SRL}

Simula is an internationally-leading Norwegian research institute in the key
ICT areas: communication systems, scientific computing and software
engineering. Simula's research areas have been evaluated with the highest
score by international expert panels in several national evaluations.

Dedicated to tackling scientific challenges with long-term impact and of
genuine importance to real life, Simula offers an environment that emphasises
and promotes basic research. This translates into numerous projects funded by
the EU, Norwegian government or regional institutions, that Simula was
involved in. In 2017, it successfully concluded Norwegian Centre of Excellence
for Biomedical Computing and is currently hosting the Centre for
Research-based Innovation, Certus. In addition, Simula is deeply involved in
research education with 35 PhD students, 40 master's students, and 20
postdoctoral fellows supervised annually; and application-driven innovation
and commercialisation, where it owns parts of 16 start-up companies with 110
employees.

The Department for Numerical Analysis and Scientific Computing (SCAN) aims to
develop mathematical methods and scientific tools to reach new understanding
of complex physical processes. It targets fundamental medical and industrial
problems where new insights from mathematical modelling can advance today's
knowledge. The department has hosted a ten-year Norwegian Centre of Excellence
in Biomedical Computing (2007-2017), one of the most prestigious research
environments in Norway, targeting ambitious and groundbreaking research. The
department received top scores in all six evaluations carried out by the
Research Council of Norway and is running a multitude of national and
international research projects, including one ERC Starter Grant project.

\subsubsection*{Curriculum vitae}
% Curriculum of the personnel at this institution

\begin{participant}[type=PI,PM=28,gender=male]{Benjamin Ragan-Kelley}
  % PM=YYY:
  % A fair evaluation of the number of months you will be
  % spending on this specific project along the four years.
  % Typical numbers:
  % - full time hired personnel: 48 months
  % - lead PI or proposal coordinator: 8-12 months
  % - PI: 4-5 months
  % - participant: 2-6 months

  % salary=ZZZ:
  % Approximate monthly gross salary (in term of total cost for the
  % employer). This is optional. If you are uncomfortable having this
  % information in a public file, you can alternatively send the
  % information to Eugenia Shadlova, or to your institution
  % leader/manager if he is willing to fill in himself the budget
  % forms on the eu portal.

Benjamin Ragan-Kelley is one of the core maintainers and developers
of the Jupyter and IPython projects, and currently leads the JupyterHub
and BinderHub development teams.
He has been a contributor to these projects since 2006,
prior to the establishment of Jupyter as a separate project from IPython.
He is an expert in all levels of Jupyter development,
especially the aspects of deploying Jupyter-based services,
which is the focus of this proposal.
Benjamin will lead \TheProject.

Beyond Jupyter, Benjamin has contributed widely to open source software,
especially in the scientific Python community.
He is a maintainer of numerous scientific packages
in the conda-forge package management system,
building packages used widely in education and research,
such as PETSc, MPICH, and FEniCS.

Benjamin is a Research Engineer in the department of Scientific Computing and Numerical Analysis
at Simula Research Laboratory in Oslo, Norway,
where his primary responsibility is developing and maintaining the Jupyter software ecosystem,
as well as supporting research scientists in diverse fields,
including biomedical computing.

Prior to his current position at Simula,
Benjamin received his Bachelor's degree \textit{Magne cum Laude} in Engineering Physics in 2007
from Santa Clara University and his PhD in Applied Science and Technology
from the University of California, Berkeley in 2013.
He worked as a postdoctoral fellow at Simula Research Laboratory prior
to becoming a permanent Research Engineer.
He was honored along with the rest of the Jupyter steering council
with the 2017 ACM Software System Award for Jupyter.


\end{participant}

%%% Local Variables:
%%% mode: latex
%%% TeX-master: "../proposal"
%%% End:


\begin{participant}[type=PM,PM=24,gender=female]{Katarina Subakova}
  % PM=YYY:
  % A fair evaluation of the number of months you will be
  % spending on this specific project along the four years.
  % Typical numbers:
  % - full time hired personnel: 48 months
  % - lead PI or proposal coordinator: 8-12 months
  % - PI: 4-5 months
  % - participant: 2-6 months

  % salary=ZZZ:
  % Approximate monthly gross salary (in term of total cost for the
  % employer). This is optional. If you are uncomfortable having this
  % information in a public file, you can alternatively send the
  % information to Eugenia Shadlova, or to your institution
  % leader/manager if he is willing to fill in himself the budget
  % forms on the eu portal.

Katarina Subakova has 9 years of experience within the EU research agenda. Currently, she holds the position of EU Funding Manager at Simula Research Laboratory in Oslo, Norway, where her responsibilities include financial and administrative management of all H2020 projects, supporting scientist in identifying funding and developing proposals, and identifying future research challenges. In addition, she serves as an external consultant to Telenor Norway for all projects funded under Horizon 2020 scheme. Prior to her current position, she headed the Horizon 2020 Helpdesk service offered by the European Commission. 

She holds the IAPP certification on GDPR. 

Katarina will lead the administrative management of the \TheProject and she will act as the project's DPO.

\end{participant}

%%% Local Variables:
%%% mode: latex
%%% TeX-master: "../proposal"
%%% End:


\begin{participant}[PM=72, type=R]{NN}

We will hire two postdoctoral-level research engineers to carry out the work
at Simula, under the leadership of and together with Dr. Ragan-Kelley.  The
fellow will have a background in computational science, combined with IPython
and Jupyter Notebook experience, and past experience of software engineering.
An ideal candidate will also have good communication skills and team working
abilities, and in particular interest and skill in the development and
operation of software services to best support this part of the project.

\end{participant}


\subsubsection*{Publications, products, achievements}

\begin{compactenum}
\item 2017 ACM Software System Award for Jupyter
\item M. Bussonier, J. Forde, J. Freeman, B. Granger, T. Head, C. Holdgraf, K.
  Kelley, G. Nalvarte, A. Osheroff, M. Pacer et al. Binder 2.0 - Reproducible,
  interactive, sharable environments for science at scale In Python in Science
  ConferenceProceedings of the 17th Python in Science Conference. Austin,
  Texas: SciPy, 2018.
\item J. Forde, T. Head, C. Holdgraf, Y. Panda, G. Nalvarte, M. Pacer, F.
  Perez, B. Ragan-Kelley and E. Sundell. Reproducible Research Environments
  with Repo2Docker In ICML 2018 Reproducible Machine Learning. ICML, 2018.
\item T. Kluyver, B. Ragan-Kelley, F. Perez, B. Granger, M. Bussonier, J.
  Frederic, K. Kelley, J. Hamrick, J. Grout, S. Corlay et al. Jupyter
  Notebooks: a publishing format for reproducible computational workflows In
  20th International Conference on Electronic Publishing. IOS Press, 2016.

\end{compactenum}

\subsubsection*{Previous projects or activities}

\begin{compactenum}
\item OpenDreamKit -
\item Jupyter - collaboration with UC Berkeley, Cal Poly, funded by Gordon \&
  Betty Moore Foundation, Alfred P. Sloan Foundation, and Helmsley Trust
\item Binder - collaboration with UC Berkeley, funded by Gordon \& Betty Moore
  Foundation
\end{compactenum}

\subsubsection*{Significant infrastructure}

The fully owned Simula subsidiary Simula Innovation handles pre-commercial
innovation projects, creation and follow-up of company spin-offs, and general
support for entrepreneurs.

\end{sitedescription}



%KEY-MORE-TODOS


%%% Local Variables:
%%% mode: latex
%%% TeX-master: "../proposal"
%%% End:

%  LocalWords:  sitedescription Simula Simula commercialisation Certus subsubsection Logg
%  LocalWords:  Mardal Funke Rognes Sci Comput Langtangen FEniCS Aln ae lgaard vspace
%  LocalWords:  TOWRITE emphasises organised
