\begin{sitedescription}{SRL}

Dedicated to tackling scientific challenges with long-term impact and of
genuine importance to real life, Simula Research Laboratory (Simula) offers an
environment that emphasises and promotes basic research. At the same time, we
are deeply involved in research education and application-driven innovation and
commercialisation.

Simula was established as a non-profit, limited company in 2001, and is fully
owned by the Norwegian Ministry of Education and Research. Its research is
funded through competitive grants from national funding agencies and the EC,
research contracts with industry, and a basic allowance from the state.
Simula’s operations are conducted in a seamless integration with the two
subsidiaries Simula School of Research and Innovation and Simula Innovation.

At its outset, the laboratory was given the mandate of becoming an
internationally leading research institution within select fields in
information and communications technology. These fields are (i) communication
systems, including cyber-security; (ii) scientific computing, aiming at fast
and reliable solutions of mathematical models in biomedicine and geoscience;
and (iii) software engineering, focusing on testing and verification of
mission-critical software systems, and on planning and cost estimation of large
software development projects. Recent evaluations state that Simula has met its
challenge and is an acknowledged contributor to top-level research in its focus
areas. Specifically, in the 2012 national evaluation of ICT research organised
by the Research Council of Norway and conducted by an international expert
panel, Simula received the highest average score (4.67) on a 1-5 scale among
all evaluated institutions. In comparison, the national average was 3.38. Only
five of the 62 research groups evaluated were awarded the top grade (5), and
two of these five groups are located at Simula. In the 2017 in-depth evaluation
of Simula, all three research areas were evaluated as excellent.

Simula has hosted one Norwegian Centre of Excellence, Center for Biomedical
Computing (2007-2017), and is currently hosting one Norwegian Centre for
Research-based Innovation, Certus (2011-2018). In addition, we participate as
research partner in two other Centres for Research-based Innovation, Centre for
Cardiological Innovation (2011-2018), hosted by Oslo University Hospital and
Centre for Scalable Data Access (2015-2022), hosted by the University of Oslo.
These two centre-oriented schemes are the most prestigious funding instruments
offered by the Research Council of Norway.

Recently, the Norwegian Ministry of Transport and Communications promoted one
of our leading research activities to national status as the Centre for
Resilient Networks and Applications, with a specific mandate to monitor the
performance and state of the national telecommunication infrastructure. The
ministry is also funding the Research Centre for Information and Communication
Security, which is a joint enterprise between Simula and the University of
Bergen. Since 2005, Simula has also developed a comprehensive industrial R\&D
collaboration with Norway’s largest commercial company, Statoil, in the area of
geological modelling for oil and gas exploration. These major initiatives are
complemented by a significant number of research grants from the Research
Council of Norway and the H2020 work programmes, each typically of 2-3 years
duration.

\subsubsection*{Education}

Ever since the start-up of Simula, education has been an important part of the
laboratory’s operations. Since 2007, educational activities have been anchored
in the subsidiary Simula School of Research and Innovation (SSRI), which is
co-owned by Simula, Statoil, the Municipality of Bærum, Telenor, SINTEF, and
the Norwegian Computing Center. Since SSRI is not a degree-awarding
institution, our students are also enrolled in degree programs at collaborating
universities in Norway and abroad. Today, about 35 PhD students, 40 master’s
students, and 20 postdoctoral fellows are supervised at Simula annually. Since
2001, more than 100 PhD students and 360 master’s students have completed their
degrees at Simula.

In addition to overseeing an efficient scheme for supervision of its enrolled
students, SSRI complements the university partners’ traditional course
portfolio with training aimed at increasing the students’ transferable skills.
In particular, SSRI offers advanced training in effective oral and written
dissemination of research, in close collaboration with leading communication
expertise at Pennsylvania State University in the United States. SSRI also
offers a modular program for professional development, including leadership
training.

\subsubsection*{Innovation}

Simula has a strong track record in developing innovative applications in
collaboration with industry partners. The fact that all three research areas at
Simula host or serve as main partner in a national Centre for Research-based
Innovation manifests this position. These three centres involve major
international industrial partners, such as GE Healthcare, Medtronic, ABB, IBM,
and Cisco, as well as several SMEs.

Simula has since its inception been strongly committed to taking research
results one step further and contribute to new and innovative solutions to
important problems. Currently, Simula holds significant portions of the shares
in 14 commercial companies. Moreover, Simula runs a start-up facility for
entrepreneurs (the Simula Garage), which currently counts more than forty
projects and is committed to continuously supporting and fostering promising
and innovative solutions with commercial potential. The fully owned subsidiary
Simula Innovation handles pre-commercial innovation projects, creation and
follow-up of company spin-offs, and general support for entrepreneurs.

\subsubsection*{Research ethics and diversity}

All employees at Simula are obliged to respect the Simula Code of Ethics. In
addition to organisational measures to prevent conflicts of interest and
corruption, to maintain confidentiality, and to support a positive and
all-including working environment, this charter states Simula’s adoption of
international standards for research ethics. In particular, Simula’s employees
shall comply with the guidelines issued by the Norway’s National Committee for
Research Ethics in Science and Technology. In addition, Simula’s publication
guidelines are based on the Vancouver Convention.

Embracing diversity enhances an organisation's range of experience, ideas and
creativity. Diversity is in all its aspects a highly respected value in the
Simula culture. Currently, Simula’s workforce is composed of about 150 people
spanning more than 30 countries of origin. In total, almost 60\% of the
employees have their background from a country abroad, contributing to a
multicultural and vibrant research environment. Gender is one important
dimension in diversity, and close to 30\% of the scientific positions is filled
by females. This ratio is even higher for recruitment positions (PhD and
postdoctoral levels). Simula's achievements for improving the gender balance
have been noted nationally, and the Norwegian Ministry of Education and
Research awarded Simula the annual Gender Balance Prize for 2013. This prize
includes a 2 million NOK (220,000 EUR) grant to be used for actions supporting
gender balance. Simula has signed the Code of Best Practices for Women in ICT,
which is part of the Digital Agenda for Europe.

\subsubsection*{Curriculum vitae}
% Curriculum of the personnel at this institution

\begin{participant}[type=PI,PM=28,gender=male]{Benjamin Ragan-Kelley}
  % PM=YYY:
  % A fair evaluation of the number of months you will be
  % spending on this specific project along the four years.
  % Typical numbers:
  % - full time hired personnel: 48 months
  % - lead PI or proposal coordinator: 8-12 months
  % - PI: 4-5 months
  % - participant: 2-6 months

  % salary=ZZZ:
  % Approximate monthly gross salary (in term of total cost for the
  % employer). This is optional. If you are uncomfortable having this
  % information in a public file, you can alternatively send the
  % information to Eugenia Shadlova, or to your institution
  % leader/manager if he is willing to fill in himself the budget
  % forms on the eu portal.

Benjamin Ragan-Kelley is one of the core maintainers and developers
of the Jupyter and IPython projects, and currently leads the JupyterHub
and BinderHub development teams.
He has been a contributor to these projects since 2006,
prior to the establishment of Jupyter as a separate project from IPython.
He is an expert in all levels of Jupyter development,
especially the aspects of deploying Jupyter-based services,
which is the focus of this proposal.
Benjamin will lead \TheProject.

Beyond Jupyter, Benjamin has contributed widely to open source software,
especially in the scientific Python community.
He is a maintainer of numerous scientific packages
in the conda-forge package management system,
building packages used widely in education and research,
such as PETSc, MPICH, and FEniCS.

Benjamin is a Research Engineer in the department of Scientific Computing and Numerical Analysis
at Simula Research Laboratory in Oslo, Norway,
where his primary responsibility is developing and maintaining the Jupyter software ecosystem,
as well as supporting research scientists in diverse fields,
including biomedical computing.

Prior to his current position at Simula,
Benjamin received his Bachelor's degree \textit{Magne cum Laude} in Engineering Physics in 2007
from Santa Clara University and his PhD in Applied Science and Technology
from the University of California, Berkeley in 2013.
He worked as a postdoctoral fellow at Simula Research Laboratory prior
to becoming a permanent Research Engineer.
He was honored along with the rest of the Jupyter steering council
with the 2017 ACM Software System Award for Jupyter.


\end{participant}

%%% Local Variables:
%%% mode: latex
%%% TeX-master: "../proposal"
%%% End:


\begin{participant}[type=PM,PM=24,gender=female]{Katarina Subakova}
  % PM=YYY:
  % A fair evaluation of the number of months you will be
  % spending on this specific project along the four years.
  % Typical numbers:
  % - full time hired personnel: 48 months
  % - lead PI or proposal coordinator: 8-12 months
  % - PI: 4-5 months
  % - participant: 2-6 months

  % salary=ZZZ:
  % Approximate monthly gross salary (in term of total cost for the
  % employer). This is optional. If you are uncomfortable having this
  % information in a public file, you can alternatively send the
  % information to Eugenia Shadlova, or to your institution
  % leader/manager if he is willing to fill in himself the budget
  % forms on the eu portal.

Katarina Subakova has 9 years of experience within the EU research agenda. Currently, she holds the position of EU Funding Manager at Simula Research Laboratory in Oslo, Norway, where her responsibilities include financial and administrative management of all H2020 projects, supporting scientist in identifying funding and developing proposals, and identifying future research challenges. In addition, she serves as an external consultant to Telenor Norway for all projects funded under Horizon 2020 scheme. Prior to her current position, she headed the Horizon 2020 Helpdesk service offered by the European Commission. 

She holds the IAPP certification on GDPR. 

Katarina will lead the administrative management of the \TheProject and she will act as the project's DPO.

\end{participant}

%%% Local Variables:
%%% mode: latex
%%% TeX-master: "../proposal"
%%% End:


\begin{participant}[PM=72, type=R]{NN} We will hire two postdoctoral-level
  research engineers to carry out the work at Simula, under the leadership of
  and together with Dr. Ragan-Kelley.  The fellow will have a background in
  computational science, combined with IPython and Jupyter Notebook experience,
  and past experience of software engineering.  An ideal candidate will also
  have good communication skills and team working abilities, and in particular
  interest and skill in the development and operation of software services to
best support this part of the project.
\end{participant}


\subsubsection*{Publications, products, achievements}

\begin{compactenum}
\item 2017 ACM Software System Award for Jupyter
\item M. Bussonier, J. Forde, J. Freeman, B. Granger, T. Head, C. Holdgraf, K.
  Kelley, G. Nalvarte, A. Osheroff, M. Pacer et al. Binder 2.0 - Reproducible,
  interactive, sharable environments for science at scale In Python in Science
  ConferenceProceedings of the 17th Python in Science Conference. Austin,
  Texas: SciPy, 2018.
\item J. Forde, T. Head, C. Holdgraf, Y. Panda, G. Nalvarte, M. Pacer, F.
  Perez, B. Ragan-Kelley and E. Sundell. Reproducible Research Environments
  with Repo2Docker In ICML 2018 Reproducible Machine Learning. ICML, 2018.
\item T. Kluyver, B. Ragan-Kelley, F. Perez, B. Granger, M. Bussonier, J.
  Frederic, K. Kelley, J. Hamrick, J. Grout, S. Corlay et al. Jupyter
  Notebooks: a publishing format for reproducible computational workflows In
  20th International Conference on Electronic Publishing. IOS Press, 2016.

\end{compactenum}

\subsubsection*{Previous projects or activities}

\begin{compactenum}
\item OpenDreamKit -
\item Jupyter - collaboration with UC Berkeley, Cal Poly, funded by Gordon \&
  Betty Moore Foundation, Alfred P. Sloan Foundation, and Helmsley Trust
\item Binder - collaboration with UC Berkeley, funded by Gordon \& Betty Moore
  Foundation
\end{compactenum}

\subsubsection*{Significant infrastructure}

The fully owned Simula subsidiary Simula Innovation handles pre-commercial
innovation projects, creation and follow-up of company spin-offs, and general
support for entrepreneurs.

\end{sitedescription}



%KEY-MORE-TODOS


%%% Local Variables:
%%% mode: latex
%%% TeX-master: "../proposal"
%%% End:

%  LocalWords:  sitedescription Simula Simula commercialisation Certus subsubsection Logg
%  LocalWords:  Mardal Funke Rognes Sci Comput Langtangen FEniCS Aln ae lgaard vspace
%  LocalWords:  TOWRITE emphasises organised
